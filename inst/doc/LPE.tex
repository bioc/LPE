% -*- mode: noweb; noweb-default-code-mode: R-mode; -*-
%\VignetteIndexEntry{LPE test for microarray data with small number of replicates}
%\VignetteKeywords{Local pooled error, replicates}
%\VignetteDepends{LPE}
%\VignettePackage{LPE}
%documentclass[12pt, a4paper]{article}
\documentclass[12pt]{article}

\usepackage{amsmath,pstricks}
\usepackage{hyperref}
\usepackage[authoryear,round]{natbib}

\textwidth=6.2in
\textheight=8.5in
%\parskip=.3cm
\oddsidemargin=.1in
\evensidemargin=.1in
\headheight=-.3in

\newcommand{\scscst}{\scriptscriptstyle}
\newcommand{\scst}{\scriptstyle}
\newcommand{\Rfunction}[1]{{\texttt{#1}}}
\newcommand{\Robject}[1]{{\texttt{#1}}}
\newcommand{\Rpackage}[1]{{\textit{#1}}}

\author{Nitin Jain, Michael O Connell, and Jae K. Lee}
\usepackage{/usr/lib/R/share/texmf/Sweave}
\begin{document}
\title{LPE test for microarray data with a small number of replicates}

\maketitle
\tableofcontents

\section{Introduction}
The \Rpackage{LPE} package describes local-pooled-error (LPE) 
test for identifying significant differentially expressed 
genes in microarray experiments. Local pooled error test is
 especially useful when the number of replicates is low (2-3)
\cite{Jain:2003}.
 LPE estimation is based on pooling errors within
genes and between replicate arrays for genes in which expression
values are similar. This is motivated by the observation that
errors between duplicates vary as a function of the average gene
expression intensity and by the fact that many gene expression
studies are implemented with a limited number of replicated arrays
\cite{Chen:1997}.

Step by step analysis is presented in Section \ref{sec:analysis} 
using data from a 6-chip oligonucleotide microarray study of a mouse immune response study.

Details of methodology and application of Local Pooled Error (LPE) test 
can be obtained from the LPE paper, published in Bioinformatics
\cite{Jain:2003}.


\section{Mouse Immune Response Study dataset}
\label{sec:analysis}
Load the library
\begin{verbatim}

> set.seed(0) # To have reproducible results
> library(LPE) 

> data(Ley)

> dim(Ley)

[1] 12488	7

> Ley[1:3,]
               ID   c1   c2   c3     t1     t2     t3
1  AFFX-MurIL2_at 16.0 14.1 19.3 2782.7 2861.3 2540.2
2 AFFX-MurIL10_at 22.7  6.9 28.2   18.6   12.7    7.5
3  AFFX-MurIL4_at 33.9 17.1 23.9   24.9   25.2   24.9

> Ley[,2:7] <- preprocess(Ley[,2:7], data.type = "MAS5")

> Ley[1:3,]

            ID       c1       c2       c3        t1        t2        t3
1  AFFX-MurIL2_at 4.058556 4.077996 4.419651 11.681194 11.668376 11.528191
2 AFFX-MurIL10_at 4.563176 3.058342 4.960576  5.125427  4.303960  3.707980
3  AFFX-MurIL4_at 5.141769 4.327081 4.718751  5.484593  5.255005  5.330675


\end{verbatim}


Mouse immune response study was conducted by 
\href{http://hsc.Virginia.EDU/medicine/basic-sci/biomed/ley/}{Dr. Klaus Ley}, Univeristy of Virginia. Three replicates of Affymetrix oligonucleotide chips per condition were used. Based on M vs A sctater plot matrix, IQR normalization was performed, so that interquartile ranges on all chips are set to their widest range. It is performed by multiplying by a scaling factor. Note that this is a simple constant-scale \& location normalization step. Finally log based 2 transformation was done. 
Replcates of Naive condition are named as c1, c2, c3 and those of 
Actiavted condition are named as t1, t2 and t3 respectively.

\vspace{0.5in}

\noindent Remove the control spots
\begin{verbatim}

> Ley <- Ley[substring(Ley$ID,1,4) !="AFFX",] 
	
> dim(Ley)

[1] 12422	7

> Ley[1:3,]

           ID        c1        c2        c3        t1        t2        t3
67   92539_at 11.999273 12.253685 12.398052 11.924385 12.042756 11.824377
68 92540_f_at  8.948516  9.034942  8.674348 11.284850 11.323132 11.289058
69   92541_at  6.242440  6.223671  6.185748  5.866883  5.775228  6.430501

\end{verbatim}
%\\
\noindent Calculate the baseline error distribution of Naive contdition, which returns a dataframe of A vs M for selected number of bins (= 1/q), 
where q = quantile. 

\begin{verbatim}

> var.Naive <- baseOlig.error(Ley[,2:4],q=0.01)

> dim(var.Naive)

[1] 12422     2

> var.Naive[1:3,]

           A      var.M
67 12.253685 0.04541592
68  8.948516 0.05166602
69  6.223671 0.24738118

\end{verbatim}
\noindent Similarly calculate the base-line distribution of Activated condition:

\begin{verbatim}

> var.Activated <- baseOlig.error(Ley[,5:7], q=0.01)

> dim(var.Activated)

[1] 12422     2

> var.Activated[1:3,]

        A      var.M
67 11.924385 0.01296899
68 11.289058 0.01884620
69  5.866883 0.27232983

\end{verbatim}

\noindent Calculate the lpe variance estimates as described above. The function {\it lpe} takes the first two arguments as the replicated data, next two arguments as the baseline distribution of the replicates calculated from the {\it baseOlig.error} function, Gene IDs as probe.set.name. Adjustment for multiple comparison is applied using Bioconductor's multtest package (Dudoit et. al.)

\begin{verbatim}
> lpe.val <- data.frame(lpe(Ley[,5:7], Ley[,2:4], var.Activated, var.Naive,
	      probe.set.name=Ley$ID))

> lpe.val <- round(lpe.val, digits=2)

> dim (lpe.val)

[1] 12422	13

> lpe.val[1:3,]

          x.t1  x.t2  x.t3 median.1 std.dev.1  y.c1  y.c2  y.c3 median.2
92539_at   11.92 12.04 11.82    11.92      0.11 12.00 12.25 12.40    12.25
92540_f_at 11.28 11.32 11.29    11.29      0.14  8.95  9.03  8.67     8.95
92541_at    5.87  5.78  6.43     5.87      0.52  6.24  6.22  6.19     6.22

           std.dev.2 median.diff pooled.std.dev z.stats
92539_at        0.21       -0.33           0.17   -1.88
92540_f_at      0.23        2.34           0.19   12.18
92541_at        0.50       -0.36           0.52   -0.68

  
\end{verbatim}

\noindent Doing FDR correction

\begin{verbatim}

> fdr.BH <- fdr.adjust(lpe.val, adjp="BH")

> dim(fdr.BH)

[1] 12422    2

> round(fdr.BH[1:4, ],2)

     FDR z.real
7915   0  26.66
5557   0  24.68
344    0  24.22
5985   0  24.22

\end{verbatim}

\noindent Resampling based FDR adjustment takes a while to run, and
returns the critical z-values and corresponding FDR. 

\begin{verbatim}

> fdr.2 <- fdr.adjust(lpe.val, adjp="resamp", iterations=2)

iteration number 1 is in progress
iteration number 1 finished
iteration number 2 is in progress
iteration number 2 finished
Computing FDR...

> fdr.2

     target.fdr z.critical
 [1,]      0.001       3.54
 [2,]      0.010       2.69
 [3,]      0.020       2.37
 [4,]      0.030       2.17
 [5,]      0.040       2.05
 [6,]      0.050       1.94
 [7,]      0.060       1.85
 [8,]      0.070       1.76
 [9,]      0.080       1.69
[10,]      0.090       1.63
[11,]      0.100       1.56
[12,]      0.150       1.31
[13,]      0.200       1.12
[14,]      0.500       0.25


\end{verbatim}


\noindent Note that above table may differ slightly due to generation 
of 'NULL distribution' by resampling. For each target.fdr, we can 
note critical z-value,
above which all genes are considered significant.

\section{Discussion}
\label{sec:discussion}
Using our LPE approach, the sensitivity of detecting subtle
expression changes can be dramatically increased and differential
gene expression patterns can be identified with both small
false-positive and small false-negative error rates. This is
because, in contrast to the individual gene's error variance, the
local pooled error variance can be estimated very accurately.
\vspace{0.25 in}

\textbf{Acknowledgments}. We wish to acknowledge the following
colleagues: P. Aboyoun, J. Betcher, D Clarkson, J. Gibson, A.
Hoering, S. Kaluzny, L. Kannapel, D. Kinsey, P. McKinnis,
D. Stanford, S. Vega and H. Yan.



\bibliography{LPE}

\end{document}





